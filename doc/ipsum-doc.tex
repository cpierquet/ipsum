% !TeX TXS-program:compile = txs:///pdflatex

\documentclass[11pt,a4paper]{ltxdoc}
\usepackage{ipsum}
\usepackage{fancyvrb}
\usepackage{fancyhdr}
\usepackage{fontawesome5}
\fancyhf{}
\renewcommand{\headrulewidth}{0pt}
%\rhead{\sffamily\small\affloetalab[Legende]}
\lfoot{\sffamily\small [ipsum]}
\cfoot{\sffamily\small - \thepage{} -}
\rfoot{\hyperlink{matoc}{\small\faArrowAltCircleUp[regular]}}
\usepackage{hologo}
\providecommand\tikzlogo{Ti\textit{k}Z}
\providecommand\TeXLive{\TeX{}Live\xspace}
\providecommand\PSTricks{\textsf{PSTricks}\xspace}
\let\pstricks\PSTricks
\let\TikZ\tikzlogo

\usepackage{hyperref}
\urlstyle{same}
\hypersetup{pdfborder=0 0 0}
\usepackage[margin=2cm]{geometry}
\setlength{\parindent}{0pt}
\def\TPversion{0.1.1}
\def\TPdate{09/11/2023}
\usepackage{tcolorbox}
\tcbuselibrary{skins,hooks}
\usepackage{soul}
\sethlcolor{lightgray!25}
\NewDocumentCommand\MontreCode{ m }{%
	\hl{\vphantom{\texttt{pf}}\texttt{#1}}%
}
\NewDocumentCommand\GenSample{ O{#2} m }{%
	%\hfill\textbf{\textsf{LANG=#1}}\par\smallskip
	\hrule\par\smallskip
	\ipsum<Type=sent,Lang=#2>\par\smallskip
	\hrule\par\smallskip
	\ipsum<Type=sent,Lang=#2>[5-6]\par\smallskip
	\hrule\par\smallskip
	\ipsum<lang=#2>[2]\par\smallskip
	\hrule\par\smallskip
	\ipsum<lang=#2>[1-3]\par\smallskip
	\hrule\par\smallskip
	\ipsum<lang=#2>[6-7]\par\smallskip
	\hrule\par\smallskip
	\ipsum<Type=enum,Lang=#2>[2-4]\par\smallskip
	\hrule\par\smallskip
	\ipsum<Type=item,Lang=#2>[8-9]\par\smallskip
	\hrule\par\smallskip
}

\begin{document}

\pagestyle{fancy}

\thispagestyle{empty}

\begin{center}
	\begin{minipage}{0.88\linewidth}
	\begin{tcolorbox}[colframe=yellow,colback=yellow!15]
		\begin{center}
			\begin{tabular}{c}
				{\Huge \texttt{ipsum}}\\
				\\
				{\LARGE Creating "Dummy Text"} \\
				{\LARGE with multilingual support.} \\
				\\
				{\small \texttt{Version \TPversion{} -- \TPdate}}
		\end{tabular}
		\end{center}
	\end{tcolorbox}
\end{minipage}
\end{center}

\begin{center}
	\begin{tabular}{c}
	\texttt{Cédric Pierquet}\\
	{\ttfamily c pierquet -- at -- outlook . fr}\\
	\texttt{\url{https://github.com/cpierquet/ipsum}} \\
	\\
	\texttt{Thanks to \url{https://ipsum.one/} for the paragraphs, with free license !}
\end{tabular}
\end{center}

\hrule

\vfill

\begin{tcolorbox}[colframe=lightgray,colback=lightgray!5]
\ipsum<Lang=EN>[6]
\end{tcolorbox}

\begin{tcolorbox}[colframe=lightgray,colback=lightgray!5]
\ipsum<Lang=LAT>[4]
\end{tcolorbox}

\begin{tcolorbox}[colframe=lightgray,colback=lightgray!5]
\ipsum<Lang=FR>[2]
\end{tcolorbox}

\begin{tcolorbox}[colframe=lightgray,colback=lightgray!5]
\ipsum<Type=enum,Lang=NL>[6-9]
\end{tcolorbox}

\begin{tcolorbox}[colframe=lightgray,colback=lightgray!5]
\ipsum<Type=item,Lang=ES>[1-3]
\end{tcolorbox}

\vfill~

\pagebreak

\phantomsection

\hypertarget{matoc}{}

\tableofcontents

\vspace*{5mm}

\hrule

\vspace*{5mm}

\section{Introduction}

\subsection{Description}

With this package you can create dummy text, in several languages.

Use \cmd{\ipsum} to get some paragraphs/sentences/lists, with optional languages and optional numbers.

Details can be found at \url{https://ipsum.one/}.

\subsection{Loading}

To load the package, simply use :

\begin{quote}
\begin{verbatim}
\usepackage{ipsum}
\end{verbatim}
\end{quote}

\vfill

\subsection{History}

\begin{quote}
\begin{verbatim}
0.1.0 : Fix typo in doc
0.1.0 : Initial version
\end{verbatim}
\end{quote}

\pagebreak

\section{The macros}

\subsection{General usage}

In order to print one or several paragraphs/sentences/lists, the command is :

\begin{quote}
\begin{verbatim}
\ipsum<Lang=...,Type=...>[range]
\end{verbatim}
\end{quote}

Available \textsf{Lang} are :

\begin{multicols}{4}
	\begin{itemize}
		\item \texttt{LAT} : Latin (def.)
		\item \texttt{EN~} : English
		\item \texttt{FR~} : French
		\item \texttt{DE~} : German
		\item \texttt{ES~} : Spanish
		\item \texttt{PT~} : Portuguese
		\item \texttt{IT~} : Italian
		\item \texttt{NL~} : Dutch
	\end{itemize}
\end{multicols}

Available \textsf{Type} are :

\begin{multicols}{2}
	\begin{itemize}
		\item \texttt{par~} : paragraphs (def.)
		\item \texttt{sent} : sentences
		\item \texttt{enum} : enumerated list
		\item \texttt{item} : itemized list
	\end{itemize}
\end{multicols}

The \textsf{range} can be given (by default it's \texttt{1}) :

\begin{itemize}
	\item by one number, between \texttt{1} and \texttt{7} ;
	\item by two numbers, \texttt{a-b} (with $\mathtt{1 \leq a < b \leq 7}$ for paragraphs, and $\mathtt{1 \leq a < b \leq 9}$ for others).
\end{itemize}

\subsection{Samples}

Each language is illustrated with :

\begin{quote}
\begin{verbatim}
\ipsum<Type=sent,Lang=...>        %the first sentence
\ipsum<Type=sent,Lang=...>[5-6]   %the 5/6th sentences
\ipsum<Lang=...>[2]               %the second paragraph
\ipsum<Lang=...>[1-3]             %the first three
\ipsum<Lang=...>[6-7]             %the last two
\ipsum<Type=enum,Lang=...>[2-4]   %enumlist with 2/3/4th sentences
\ipsum<Type=item,Lang=...>[8-9]   %itemist with last two sentences
\end{verbatim}
\end{quote}

\pagebreak

\subsubsection{Lang = Default}

\GenSample[Default]{LAT}

\pagebreak

\subsubsection{Lang = EN}

\GenSample{EN}

\pagebreak

\subsubsection{Lang = FR}

\GenSample{FR}

\pagebreak

\subsubsection{Lang = DE}

\GenSample{DE}

\pagebreak

\subsubsection{Lang = ES}

\GenSample{ES}

\pagebreak

\subsubsection{Lang = PT}

\GenSample{PT}

\pagebreak

\subsubsection{Lang = IT}

\GenSample{IT}

\pagebreak

\subsubsection{Lang = NL}

\GenSample{NL}

\end{document}